\documentclass[12pt]{article}
\usepackage{amsmath}
\usepackage{listings}
\begin{document}
This will be a document with some information on Unix commands we have seen in class.

\section{Commands 8/24-26/22}
\begin{enumerate}
\item clear           : will clear the terminal so you can have a fresh starting point.
\item cd              : will show directory path
\item cd\verb'<path>' : cahnges your command line to be on the stated path
\item mkdir           : when given a name as a argument this creates a directory at your specified path position
\item ls              : will list all the stuff in current directory
	\begin{enumerate}
		\item -a: lists hidden files that with a "."
		\item -l: list files permission and sizes
	\end{enumerate}

\item nano            : this is a simple text editor
\item man             : this searches the manual pages for information on linux commands
\end{enumerate}


\section{New Commands 8/29/22}
\begin{enumerate}
  \item pwd               : Print Working Directory
  \item \verb'<tab>'      : Auto-Complete for whatever is presented on the path
  \item \verb'<Up Arrow> ': This puts last used command in terminal, cycle through history of used commands
  \item cp                : Copy file from one place to another
  \item rm                : remove file name
  \begin{enumerate}
    \item rm *.pdf         : removes all files with the extension of '.pdf'
    \item rm -R directory  : removes files recursively from a directory
  \end{enumerate}

\item top              : shows processes open and running on a machine as well as the process ids (PID) number, Use 'Q' to quit top
\begin{verbatim}
tunno@Savage> top -U <Lname>
\end{verbatim}
This shows your job running on savage
\item Kill: use this with PID, this ends the job on the machine
\end{enumerate}

\section{New commands 8/31/22}
Some new fun things:
\subsection{the {\bf head} and {\bf tail} commands:} 
\begin{verbatim}
tunno@savage> head -n filename
\end{verbatim}
This displays to standard out the first  $n$ lines of the specific file. Similar for the {\it tail} command

Note we can use the  \verb'>' % This is a comment in tex (Greater than symbol)
To divert information from the terminal stream to a specified file.

\subsection*{Other ways to Look at Files}
You can use {\it cat} to dump all the contents of a file to standard out. The tools:
\begin{enumerate}
\item less: file viewer(scroll with arrows)
\item more: file viewer(view pages with spacebar/tab)
\end{enumerate}
give you a little more refinement when looking at the file content. These are not editors
%Less is really better than more, as less has more features

We also talked about the {\it history} command. The history command allows you to see the $n$ comands executed in the terminal so you know what you did
 last time or creating log files.


\section*{Notes 9/2/22}
This will document using a listing inside of a document
\begin{lstlisting}[language=bash]
ssh tunno@savage.nsm.iup.edu
\end{lstlisting}
This type works for all types of code. Say you had python script you wanted to document.
\begin{lstlisting}[language=python, caption=Basic python script"]
for i in range(10):
	print("Python is better than Java")
	print("especially on Unix")

windows = 10
unix    = 11
if (windows < unix):
	print("You know its true")
\end{lstlisting}

I would like to input information or code from some other file into this document.
\lstinputlisting[language=bash]{ShortHistory.log}


\subsection*{Lets include all the text for this document:}
Here is a listing showing the \LaTeX\ markup for the document itself
\lstinputlisting[language=tex]{BasicUnixCommand.tex}


\end{document}
